%%%%%%%%%%%%%%%%%%%%%%%%%%%%%%%%%%%%%%%%%%%%%%%%%%%%%%%%%%%%%%%%%%%%%%%%%
% REV349 Sun 26 Sep 2021 09:13:27 WIB
% REV154 Thu Aug 23 11:22:02 WIB 2018
% START0 Thu Jul 26 20:01:45 WIB 2018
%%%%%%%%%%%%%%%%%%%%%%%%%%%%%%%%%%%%%%%%%%%%%%%%%%%%%%%%%%%%%%%%%%%%%%%%%

\section{Week 04: Topics}
\begin{frame}[fragile]
\frametitle{Week 04 Addressing:
Topics\footnote{Source: ACM IEEE CS Curricula 2013}}

\begin{itemize}
\item Bits, bytes, and words
\item Numeric data representation and number bases
\item Representation of records and arrays
\end{itemize}
\end{frame}


\begin{frame}[fragile]
\frametitle{Week 04 Addressing:
Learning Outcomes\footnote{Source: ACM IEEE CS Curricula 2013}}
\begin{itemize}
\item Explain why everything is data, including instructions, in computers. [Familiarity]
\item Explain the reasons for using alternative formats to represent numerical data. [Familiarity]
\item Describe the internal representation of non-numeric data, such as characters, strings, records, and arrays.  [Familiarity]
\end{itemize}

\end{frame}

