%%%%%%%%%%%%%%%%%%%%%%%%%%%%%%%%%%%%%%%%%%%%%%%%%%%%%%%%%%%%%%%%%%%%%%%%
% Beamer Presentation - LaTeX - Template Version 1.0 (10/11/12)
% This template has been downloaded from: http://www.LaTeXTemplates.com
% License: % CC BY-NC-SA 3.0 (http://creativecommons.org/)
% Modified by Rahmat M. Samik-Ibrahim

% REV005 Wed 19 Jan 2022 08:44:07 WIB
% REV004 Tue 18 Jan 2022 02:56:05 WIB
% REV003 Mon 17 Jan 2022 14:14:04 WIB
% REV002 Sat 15 Jan 2022 17:51:47 WIB
% REV001 Fri 14 Jan 2022 17:09:33 WIB
% STARTX Wed 14 Sep 2016 10:45:19 WIB
%%%%%%%%%%%%%%%%%%%%%%%%%%%%%%%%%%%%%%%%%%%%%%%%%%%%%%%%%%%%%%%%%%%%%%%%%

% PACKAGES AND THEMES 
\documentclass[xcolor=table, notheorems, hyperref={pdfpagelabels=false}]{beamer}
\input{beamer.tex}
\newcommand{\revision}{REV002 15-Jan-2022}
% w! tmptmp
% REV002 Sat 15 Jan 2022 17:51:47 WIB
% REV001 Fri 14 Jan 2022 18:00:00 WIB
% STARTS Wed 24 Aug 2016 19:34:33 WIB
%%%%%%%%%%%%%%%%%%%%%%%%%%%%%%%%%%%%%
\newcommand{\kopikopi}{\textcopyright{}2022-2022 Demo Suremo}



% XXXXXXXXXXXXXXXXXXXXXXXXXXXXXXXXXXXXXXXXXXXXXXXXXXXXXXXXXXXXXXXXXXXXXXXXXX
% The short title appears at the bottom of every slide, 
% the full title is only on the title page
% \date{\today}

\title[\kopikopi]
{BackUp dengan RSYNC dan GIT}
\author{Demo Suremo}
\institute[SdnBhd]{
Sendirian Berhad\\
\medskip
\url{https://demosuremo.github.io/demo/LaTeX/generic.pdf}
\\ \texttt{Harap cek revisi terakhir!}
}
\date{\revision}
\titlegraphic{\includegraphics[width=120mm]{demo-youtube}}

% XXXXXXXXXXXXXXXXXXXXXXXXXXXXXXXXXXXXXXXXXXXXXXXXXXXXXXXXXXXXXXXXXXXXXXXXXX
\begin{document}

\lstset{
basicstyle=\ttfamily\tiny, % \tiny \small \footnotesize 
breakatwhitespace=true,
language=C,
columns=fullflexible,
keepspaces=true,
breaklines=true,
tabsize=3, 
showstringspaces=false,
extendedchars=true}

\section{Start}
\begin{frame}
\titlepage
\end{frame}

% XXXXXXXXXXXXXXXXXXXXXXXXXXXXXXXXXXXXXXXXXXXXXXXXXXXXXXXXXXXXXXXXXXXXXXXXXX
% Throughout your presentation, if you choose to use \section{} and 
% \subsection{} commands, these will automatically be printed on 
% this slide as an overview of your presentation
\section{Agenda}
\begin{frame}{Outline}
  \frametitle{Agenda}
  \tableofcontents[sections={1-11}]
\end{frame}
% \begin{frame}
%    \frametitle{Agenda (2)}
%    \tableofcontents[sections={12-}]
% \end{frame}

% XXXXXXXXXXXXXXXXXXXXXXXXXXXXXXXXXXXXXXXXXXXXXXXXXXXXXXXXXXXXXXXXXXXXXXXXXX
% \tiny	\scriptsize \footnotesize \small \normalsize \large \Large
\section{Memperkenalkan diri}
\begin{frame}[fragile]
\frametitle{Memperkenalkan diri}
% \large(54) \small(65) \footnotesize(72) \tiny(108) 
% \begin{lstlisting}[basicstyle=\ttfamily\small]
% \begin{lstlisting}[basicstyle=\ttfamily\footnotesize]
% \begin{lstlisting}[basicstyle=\ttfamily\tiny]
\begin{figure}
\includegraphics[width=0.48\linewidth]{JPG-012}
\end{figure}
\begin{lstlisting}[basicstyle=\ttfamily\large]
Nama Saya Rahmat...  Rahmat Nama Saya...
Kalau Bukan Rahmat...  Bukan Nama Saya...
\end{lstlisting}
{\tiny
\begin{itemize}
\item \url{https://rahmatm.samik-ibrahim.vlsm.org/2016/08/panggil-saya-rahmat.html})
\item \url{https://rahmatm.samik-ibrahim.vlsm.org/2021/01/how-to-contact-rahmat-m-samik-ibrahim.html}
\end{itemize}
}
\end{frame}

% XXXXXXXXXXXXXXXXXXXXXXXXXXXXXXXXXXXXXXXXXXXXXXXXXXXXXXXXXXXXXXXXXXXXXXXXXX
\section{Ada Saja Yang Lupa Backup!}
\begin{frame}[fragile]
\frametitle{Ada Saja Yang Lupa Backup!}
\begin{figure}
\includegraphics[width=0.74\linewidth]{puss-in-boot}
\caption{Wajah Memelas Akibat Lalai Backup!}
\end{figure}
\end{frame}

% XXXXXXXXXXXXXXXXXXXXXXXXXXXXXXXXXXXXXXXXXXXXXXXXXXXXXXXXXXXXXXXXXXXXXXX
\begin{frame}[fragile]
\section{Backup menggunakan RSYNC dan GIT}
\frametitle{Backup menggunakan RSYNC dan GIT}
\begin{itemize}
\item Cara \textbf{SAYA} mem-backup dari masa ke masa.
\item (Pengguna) Ponsel Andriod (berbasis Google).
\item Berbasis GNU/Linux menggunakan RSYNC dan GIT.
\item URLs:
\begin{itemize}
\item Presentasi PDF - {\scriptsize \url{https://demosuremo.github.io/demo/LaTeX/generic.pdf}}
\item Google Account Backup - {\scriptsize \url{https://demosuremo.github.io/demo/001.html}}
\item RSYNC - {\scriptsize \url{https://demosuremo.github.io/demo/002.html}}
\item LaTeX Presentasi - {\tiny \url{https://github.com/demosuremo/demo/tree/master/LaTeX/src/}}
\item GitHub Page - \url{https://demosuremo.github.io/demo/}
\item GitHub - \url{https://github.com/demosuremo/demo/}
\item MarkDown Web (Jekyll) Listing - {\tiny \url{https://demosuremo.github.io/demo/000.html}}
\item LaTeX Examples with TarBalls - \url{https://latex.vlsm.org/}
\item Installing Jekyll On a VirtualBox  {\scriptsize - \url{https://doit.vlsm.org/005.html}}
\item GitHub Page Template - \url{https://template.vlsm.org/}
\item Debian on VirtualBox - \url{https://osp4diss.vlsm.org/}
\end{itemize}
\end{itemize}
\end{frame}

% XXXXXXXXXXXXXXXXXXXXXXXXXXXXXXXXXXXXXXXXXXXXXXXXXXXXXXXXXXXXXXXXXXXXXXX
\begin{frame}[fragile]
\section{Kilas Balik Unit Penyimpanan}
\frametitle{Kilas Balik Unit Penyimpanan}
\begin{itemize}
\item 1981: membeli disket Dysan 5.25" (360 kB) pertama seharga Rp. 7500.- (setara Rp. 300 000).
\item 1984: hard disk $\sim$ 10 MB.
\item 1990: hard disk $\sim$ 40 MB.
\item 1995: CD Read/Write 700 MB.
\item 1999: DVD Read/Write 4.7 GB
\item 2000: hard disk $\sim$ 30 GB.
\item 2001: flash disk ''key'' $\sim$ 8 MB.
\item 2010: hard disk $\sim$ 250 GB per plat.
\item 2011: flash disk ''key'' $\sim$ 64 GB.
\end{itemize}
\end{frame}

% XXXXXXXXXXXXXXXXXXXXXXXXXXXXXXXXXXXXXXXXXXXXXXXXXXXXXXXXXXXXXXXXXXXXXXX
\begin{frame}[fragile]
\section{Zaman Google}
\frametitle{Zaman Google}
\begin{itemize}
\item 1998: Google Search
\item 2004: Zaman GMAIL dimulai (2 GB)
\item 2006: Membeli YouTube
\item 2008: Android 1.0
\item 2011: Google Takeout - \url{https://takeout.google.com/}
\begin{itemize}
\item See also \url{https://demosuremo.github.io/demo/001.html}
\end{itemize}
\item 2012: Google Drive (15 GB)
\item 2013: Membeli Waze
\item 2018: Google One
\end{itemize}

\noindent\rule{\textwidth}{1pt}

\begin{itemize}
\item \href{https://support.google.com/android/answer/2819582}{Back up or restore data on your Android device}
\item Backup Manual: Email (via Thunderbird)
\item Backup Manual: Contact
\item Backup Manual: Blogspot
\end{itemize}
\end{frame}

% XXXXXXXXXXXXXXXXXXXXXXXXXXXXXXXXXXXXXXXXXXXXXXXXXXXXXXXXXXXXXXXXXXXXXXX
\begin{frame}[fragile]
\section{Directories}
\frametitle{Directories}
\begin{figure}
\includegraphics[width=0.69\linewidth]{DIRECTORIES}
\caption{DIRECTORIES}
\end{figure}
\end{frame}

% XXXXXXXXXXXXXXXXXXXXXXXXXXXXXXXXXXXXXXXXXXXXXXXXXXXXXXXXXXXXXXXXXXXXXXX
\begin{frame}[fragile]
\section{FlashDisk}
\frametitle{FlashDisk}
\begin{figure}
\includegraphics[width=0.69\linewidth]{FLASHDISK}
\caption{FLASHDISK}
\end{figure}
\end{frame}

% XXXXXXXXXXXXXXXXXXXXXXXXXXXXXXXXXXXXXXXXXXXXXXXXXXXXXXXXXXXXXXXXXXXXXXX
\begin{frame}[fragile]
\section{Test Cloud Kilat}
\frametitle{Test Cloud Kilat}
\begin{itemize}
\item XX
\end{itemize}
\end{frame}

% XXXXXXXXXXXXXXXXXXXXXXXXXXXXXXXXXXXXXXXXXXXXXXXXXXXXXXXXXXXXXXXXXXXXXXX
\end{document}
